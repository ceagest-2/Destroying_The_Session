\documentclass[12pt]{article}
\usepackage[utf8]{inputenc}
\usepackage[russian]{babel}
\usepackage{geometry}
\usepackage{titlesec}
\usepackage{enumitem}
\usepackage{xcolor}
\usepackage{hyperref}

\geometry{a4paper, margin=2cm}
\titleformat{\section}{\large\bfseries}{}{0em}{}
\titleformat{\subsection}{\bfseries}{}{0em}{}
\renewcommand{\labelitemi}{--}

\hypersetup{
    colorlinks=true,
    linkcolor=blue,
    filecolor=magenta,  
    urlcolor=cyan,
}

\title{Общие понятия \texttt{systemd}}
\author{}
\date{}

\begin{document}

\maketitle

\section*{Что такое systemd?}

\texttt{systemd} — это менеджер системы и служб для операционных систем Linux. Он запускается как процесс с PID 1 и отвечает за инициализацию всей остальной части системы.

\subsection*{Основные функции}
\begin{itemize}

\item Параллельный запуск системных служб во время загрузки (существенно ускоряет старт системы).
    \item Активация демонов по требованию.
    \item Управление службами на основе зависимостей.
    \item Контроль за жизненным циклом сервисов: запуск, остановка, перезагрузка, мониторинг.
\end{itemize}
\section*{Компоненты systemd}

\texttt{systemd} представляет собой набор базовых компонентов, управляющих различными аспектами системы:
\begin{itemize}
\item Запуск и управление службами.
    \item Монтирование файловых систем.
    \item Работа с устройствами.
    \item Управление сетью.
    \item Журналирование (через \texttt{journalctl}).
\end{itemize}
\section*{Понятие Unit}

В \texttt{systemd} все управляемые объекты называются \textbf{unit} (юнит). Они представлены конфигурационными файлами и классифицируются по типам.

\subsection*{Типы юнитов}
\begin{center}
\begin{tabular}{|l|l|p{7cm}|}
\hline
\textbf{Тип} & \textbf{Расширение} & \textbf{Описание} \\
\hline
Service unit & \texttt{.service} & Запускает и контролирует процесс. \\
\hline
Target unit & \texttt{.target} & Группа сервисов (аналог runlevel). \\
\hline
Mount unit & \texttt{.mount} & Точка монтирования. \\
\hline
Automount unit & \texttt{.automount} & Автоматическая точка монтирования. \\
\hline
Socket unit & \texttt{.socket} & Сетевой или IPC-сокет. \\
\hline
Timer unit & \texttt{.timer} & Служба времени (аналог cron). \\
\hline
Device unit & \texttt{.device} & Устройство. \\
\hline
Swap unit & \texttt{.swap} & Файл подкачки. \\
\hline
Path unit & \texttt{.path} & Мониторинг изменений в файловой системе. \\
\hline
Snapshot unit & \texttt{.snapshot} & Сохранение состояния системы. \\
\hline
Scope unit & \texttt{.scope} & Управление внешними процессами. \\
\hline
Slice unit & \texttt{.slice} & Группа процессов, организованных иерархически. \\
\hline
\end{tabular}
\end{center}

\section*{Расположение файлов юнитов}

Файлы юнитов располагаются в нескольких каталогах, приоритет которых определяется порядком:
\begin{itemize}
\item \texttt{/etc/systemd/system/} — юниты, созданные при включении systemctl \textbf{Наивысший приоритет}.
    \item \texttt{/run/systemd/system/} —  юниты, созданные во время выполнения.
    \item \texttt{/usr/lib/systemd/system/} — юниты, поставляемые с установленными пакетами. \textbf{Наименьший приоритет}.
\end{itemize}
\section*{Управление службами: systemctl}

Команда \texttt{systemctl} используется для управления системой и службами.

\subsection*{Базовые операции}
\begin{itemize}
\item \texttt{systemctl start <name.service>} — запустить службу.
    \item \texttt{systemctl stop <name.service>} — остановить службу.
    \item \texttt{systemctl restart <name.service>} — перезапустить службу.
    \item \texttt{systemctl reload <name.service>} — перезагрузить конфигурацию.
    \item \texttt{systemctl status <name.service>} — проверить статус службы.
    \item \texttt{systemctl is-active <name.service>} — проверить, активна ли служба.
    \item \texttt{systemctl list-units --type=service --all} — показать все службы.
\end{itemize}
\subsection*{Операции включения/выключения}
\begin{itemize}
\item \texttt{systemctl enable <name.service>} — включить автозапуск службы.
    \item \texttt{systemctl disable <name.service>} — отключить автозапуск.
    \item \texttt{systemctl is-enabled <name.service>} — проверить, включена ли служба.
    \item \texttt{systemctl list-unit-files --type=service} — список всех служб и их статус автозапуска.
\end{itemize}
\section*{Создание своего юнита}

Файлы пользовательских юнитов размещаются в каталоге \texttt{/etc/systemd/system/}.

Пример минимального файла службы \texttt{mybestservice.service}:

\begin{verbatim}
[Unit]
Description=MyBestService
After=network.target
Requires=mysql.service

[Service]
Type=simple
User=myunit
Group=myunit
WorkingDirectory=/work/www/myunit/current
ExecStart=/usr/local/bin/bundle exec service -C /work/www/myunit/shared/config/service.rb --daemon
ExecStop=/usr/local/bin/bundle exec service -S /work/www/myunit/shared/tmp/pids/service.state stop
Restart=always

[Install]
WantedBy=multi-user.target
\end{verbatim}

\subsection*{Секция [Unit]}
\begin{itemize}
\item \texttt{Description=} — описание службы.
    \item \texttt{After=} — указывает, после каких юнитов должна запускаться служба.
    \item \texttt{Requires=} — жесткая зависимость: если зависимый сервис не запущен, текущий не будет активирован.
    \item \texttt{Wants=} — слабая зависимость: рекомендуется, но не обязательно.
\end{itemize}
\subsection*{Секция [Service]}
\begin{itemize}
\item \texttt{Type=simple} — процесс запускается сразу (по умолчанию).
    \item \texttt{Type=forking} — процесс "разветвляется" (демон); требуется указать \texttt{PIDFile=}.
    \item \texttt{User=}, \texttt{Group=} — пользователь и группа для запуска.
    \item \texttt{WorkingDirectory=} — рабочий каталог.
    \item \texttt{Environment=} — переменные окружения.
    \item \texttt{OOMScoreAdjust=-1000} — снижает вероятность завершения процесса при нехватке памяти.
    \item \texttt{ExecStart=}, \texttt{ExecStop=}, \texttt{ExecReload=} — команды запуска, остановки и перезагрузки.
    \item \texttt{TimeoutSec=} — таймаут ожидания запуска.
    \item \texttt{Restart=always} — автоматический перезапуск при падении.
\end{itemize}
\subsection*{Секция [Install]}
\begin{itemize}
\item \texttt{WantedBy=multi-user.target} — служба будет запускаться при загрузке в многопользовательском режиме.
    \item \texttt{WantedBy=graphical.target} — для графического режима.
\end{itemize}
\end{document}