\documentclass[12pt]{article}
\usepackage[utf8]{inputenc}
\usepackage[russian]{babel}
\usepackage{geometry}
\usepackage{listings}
\usepackage{xcolor}
\usepackage{titlesec}

\geometry{a4paper, margin=2cm}
\titleformat{\section}{\large\bfseries}{}{0em}{}
\titleformat{\subsection}{\bfseries}{}{0em}{}

\definecolor{codegreen}{rgb}{0,0.6,0}
\definecolor{codegray}{rgb}{0.5,0.5,0.5}
\definecolor{codepurple}{rgb}{0.58,0,0.82}

\lstdefinestyle{bashstyle}{
    backgroundcolor=\color{white},
    commentstyle=\color{codegreen},
    keywordstyle=\color{magenta},
    numberstyle=\tiny\color{codegray},
    stringstyle=\color{codepurple},
    basicstyle=\ttfamily\small,
    breakatwhitespace=false,
    breaklines=true,
    captionpos=b,
    keepspaces=true,
    numbers=left,
    numbersep=5pt,
    showspaces=false,
    showstringspaces=false,
    showtabs=false,
    tabsize=2,
    language=bash
}

\title{Лекция №6: Bash и пакетные менеджеры}
\author{}
\date{}

\begin{document}

\maketitle

\section*{Bash}

\subsection*{Что это такое?}
\begin{itemize}
\item Bash — сценарии оболочки bash.
\end{itemize}
\subsection*{Зачем?}
\begin{itemize}
\item Простая автоматизация действий и скриптов.
    \item Доступен где есть оболочка.
    \item Простая интеграция с другими приложениями.
\end{itemize}
Bash-скрипт — это файл, содержащий последовательность команд, которые выполняются программой bash построчно. Он позволяет выполнять ряд действий, таких как переход к определённому каталогу, создание папки и запуск процесса с помощью командной строки.

Сохранив эти команды в сценарии (скрипты), вы можете повторять одну и ту же последовательность шагов несколько раз и выполнять их, запустив сценарий.

\section*{Hello World на bash}

Пример скрипта:
\begin{lstlisting}[style=bashstyle]
#!/bin/bash
echo "Hello, World!"
\end{lstlisting}

Шебанг \texttt{\#!/bin/sh} не обязателен — он может быть изменён под нужную программу, например:
\begin{lstlisting}[style=bashstyle]
#!/usr/bin/env python
\end{lstlisting}

Такой шебанг использует утилиту \texttt{env}, которая ищет исполняемый файл в переменной окружения PATH. Плюс — гибкость пути. Минус -- первый попавшийся файл может не подходить.

\section*{Переменные}

Присваивание переменных:
\begin{lstlisting}[style=bashstyle]
name="Alex"
echo $name
echo ${name}
\end{lstlisting}

\subsection*{Особые переменные}
\begin{itemize}
\item \texttt{\$@} — все аргументы скрипта.
\item \texttt{\$i} — iй аргумент скрипта (нумерация с 1).
\end{itemize}
\section*{Литералы}

Оболочка выполняет подстановку переменных, шаблонов и т.д. перед выполнением команды. Кавычки позволяют создавать литералы — строки, которые не анализируются оболочкой.
\begin{itemize}
\item Одинарные кавычки (\texttt{'}): всё внутри интерпретируется буквально.
\end{itemize}
Пример:
\begin{lstlisting}[style=bashstyle]
grep 'r.*t' /etc/passwd
\end{lstlisting}

\section*{Математические операции}

Используется конструкция \texttt{\$((...))}:
\begin{lstlisting}[style=bashstyle]
result=$((5 + 3))
echo $result  \# 8
\end{lstlisting}

\section*{Управление потоком}

\begin{lstlisting}[style=bashstyle]
if [ "$1" = "hi" ]; then
    echo "Hi!"
else
    echo "Bye!"
fi
\end{lstlisting}

\texttt{[} — это отдельная команда (синоним \texttt{test}), а не часть синтаксиса.

\section*{Логические выражения}

\subsection*{Операторы для числового сравнения}
\begin{itemize}
\item \texttt{-eq} — равно
    \item \texttt{-ne} — не равно
    \item \texttt{-lt} — меньше
    \item \texttt{-le} — меньше или равно
    \item \texttt{-gt} — больше
    \item \texttt{-ge} — больше или равно
\end{itemize}
\subsection*{Операторы проверки прав доступа}
\begin{itemize}
\item \texttt{-r} — файл доступен на чтение
    \item \texttt{-w} — файл доступен на запись
    \item \texttt{-x} — файл исполняемый
\end{itemize}
\subsection*{Операторы проверки типов файлов}
\begin{itemize}
\item \texttt{-f} — обычный файл
    \item \texttt{-d} — каталог
\end{itemize}
\section*{Циклы}

Пример цикла \texttt{for}:
\begin{lstlisting}[style=bashstyle]
for i in {1..5}; do
    echo "Iteration $i"
done
\end{lstlisting}

Пример цикла \texttt{while}:
\begin{lstlisting}[style=bashstyle]
count=1
while [ $count -le 5 ]; do
    echo $count
    count=$((count + 1))
done
\end{lstlisting}

\section*{Функции}

\begin{lstlisting}[style=bashstyle]
greet() {
    echo "Hi, $1!"
}
greet "Ann"
\end{lstlisting}

\section*{Пример использования awk}

Пример фильтрации данных:
\begin{lstlisting}[style=bashstyle]
ps aux | awk '{print $1, $2, $11}' | head -5
\end{lstlisting}

\section*{Установка пакетов}

\subsection*{Что такое пакет?}
\begin{itemize}
\item Пакет — набор команд и зависимостей для установки ПО.
    \item \texttt{.deb} — Debian и производные (Ubuntu, Mint и др.)
    \item \texttt{.rpm} — Red Hat и производные (CentOS, Fedora, OpenSUSE)
    \item \texttt{.apk} — Android
    \item \texttt{.ebuild} — Gentoo
\end{itemize}
\subsection*{Пакетные менеджеры}
\begin{itemize}
\item APT
    \item RPM
    \item YUM
    \item DNF (Dandified YUM)
    \item Pacman
\end{itemize}

\section*{Что такое репозиторий?}
Репозиторий (от англ. \textit{repository} — хранилище) — место, где хранятся и поддерживаются данные, чаще всего в виде файлов, доступных для распространения по сети.

Официальные репозитории есть у большинства дистрибутивов Linux. Также можно добавлять сторонние.


\end{document}