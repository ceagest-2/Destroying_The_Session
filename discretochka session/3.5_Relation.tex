\documentclass{article}
\usepackage{amsmath, amssymb}
\usepackage[russian]{babel}
\usepackage{mathtools}
\usepackage{hyperref}


\begin{document}

\title{Примеры отношений частичного порядка, формальное определение. Линейный порядок.
Отношение непосредственного следования и его граф (диаграмма Хассе).}

\author{\href{https://t.me/dan_the_fox}{Составлял Даня}}
\date{}
\maketitle

\paragraph{Отношение на множестве} $\\$
$\indent$ Пусть $A$ -- это произвольное множество. \textbf{Отношением} на множестве $A$ называется произвольное подмножество его декартового произведения: $R \subset A \times A$. $\\$
$\indent$ Если пара элементов $(x, y)$ содержатся в отношении $R$ ($(x, y) \in R$), то говорят, что $x$ относится к $y$ по отношению $R$ и записывают $x R y$. Запись $x R y$ эквивалентна $(x, y) \in R$.
$\\$$\indent$
Рассмотрим свойства:
\begin{enumerate}
    \item \emph{рефлексивность},  $\forall x \in A \; x R x$;
    \item \emph{транзитивность}, $x R y$ , $y R z \Rightarrow  x R z$.
    \item \emph{симметричность:} $x R y \Rightarrow y R x$;
    \item \emph{антисимметричность:} $x R y$ , $y R x \Rightarrow x = y$. Иначе, $x \neq y \Rightarrow (x, y) \notin R$ или $(y, x) \notin R$.
\end{enumerate}

\paragraph{Отношение эквивалентности} $\\$$\indent$ 
Если для отношения R выполняются свойства 1, 2, 3, то оно называется \textbf{отношением эквивалентности} и обозначается $\sim$. 

\paragraph{Нестрогое отношение частичного порядка} $\\$$\indent$ 
Если для отношения R выполняются свойства 1, 2, 4, то оно называется \textbf{частичным порядком} и обозначается $\preceq$.
$\\$$\indent$ В качестве примера можно взять обычное нестрогое неравество.

\paragraph{Линейный порядок} $\\$$\indent$ 
Пусть $\preceq $ -- частичный порядок на множестве $A$. 2 произвольных элемента $x, y \in A$ называются \emph{сравнимыми}, если $x \preceq  y$ или $y \preceq x$. Если для частичного порядка $\preceq$ любые 2 элемента сравнимы между собой, то такой частичный порядок называется \textbf{линейным порядком}.

\paragraph{Строгое отношение частичного порядка} $\\$$\indent$ 
Пусть $\preceq$ --- произвольное отношение частичного порядка на множестве $A$. Если $a, b \in A$ и $a \preceq b$, то говорят, что элемент $a$ \emph{предшествует или равен} элементу $b$, или элемент $b$ \emph{следует или равен} элементу $a$.

Пусть $(A; \preceq)$ --- частично упорядоченное множество. Если для элементов $a, b \in A$ верно $a \preceq b$ и $a \neq b$, то пишут $a \prec b$ и говорят, что элемент $a$ \emph{строго предшествует} элементу $b$, или что элемент $b$ \emph{строго следует} за элементом $a$.  
$\\$$\indent$  Такое отношение можно ввести и на всем множестве. Следует заметить, что у этого \textbf{строгого отношения частичного порядка} отсутствует рефлексивность.

\paragraph{Непосредственное следование} $\\$$\indent$ 
Если для элементов $a, b \in A$ верно $a \prec b$ и не существует такого элемента $c \in A$, что $a \prec c \prec b$ (т.е. нет промежуточного элемента между $a$ и $b$), то говорят, что элемент $a$ \emph{непосредственно предшествует} элементу $b$, или что элемент $b$ \emph{непосредственно следует} за элементом $a$. 

\paragraph{Диаграмма Хассе} $\\$$\indent$ 
Диаграмма Хассе для ЧУМ $(A; \preceq)$ --- это ориентированный граф, в котором вершинами выступают элементы $A$, а ребро от $a$ к $b$ проводится, если $a$ непосредственно предшествует $b$. 
$\\$$\indent$ 
Аналогично можно определить диаграмму с помощью транзитивной редукции --- операции, при которой из графа отношения частичного порядка удаляются ребра, которые можно восстановить из транзитивности.

\textbf{Замечание.} В курсе не определяли простой путь для ориентированного графа.

\paragraph{Отношение двухсторонней достижимости} $\\$$\indent$ 
Между вершинами u и v ориентированного графа G существует отношение двухсторонней достижимости, если существует \textbf{путь} $u \rightarrow v$ и $v \rightarrow u$. 
Оно является отношением эквивалентности.

\paragraph{Классы эквивалентности в ориентированном графе} $\\$$\indent$
Классами эквивалентности в ориентированном графе называются компоненты сильной связности --- набор вершин такой, что из любой другой можно прийти в любую другую.

\paragraph{Ациклический ориентированный граф} $\\$$\indent$
В ациклическом ориентированном графе нет замкнутных маршрутов положительной длины.

\textbf{Teорема}. Пусть дан ориентированный граф $G(V, E)$ без петель. Следующие утверждения эквивалентны:
\begin{enumerate}
    \item Граф G ациклический;
    \item Компоненты сильной связности иметь размер $1$;
    \item Существует нумерация вершин т.ч. $(u_i, u_j) \in E \Rightarrow i < j$;
\end{enumerate}
\textbf{Доказательство}. $\\$$\indent$
\begin{itemize}
\item $1 \iff 2$ очевидно из определений. $\\$$\indent$
\item $3 \Rightarrow 1$ Если бы существовал замкнутый маршрут 
\begin{center}
    $v_{i_1} \rightarrow v_{i_2} $...$ \rightarrow v_{i_n} = v_{i_1}$,
\end{center} $\indent$ где индексы вершин -- натуральные числа, то имели бы 
\begin{center}
    $i_1 < i_2 < $...$ < i_n = i_1$,
\end{center} $\indent$ Противоречие.
$\\$
$\\$$\indent$
\emph{Лемма.} Если ориентированный граф ациклический, то в нем есть вершина с нулевой исходящей степенью.
    $\\$$\indent\indent$
    \emph{Доказательство.} Так как вершин конечное число, то не существовало было был пути наибольшей длины. В таком пути все вершины буду различны, иначе образуется цикл

\item $1 \Rightarrow 3$  Докажем утверждение по индукции по количеству вершин в графе.
 $\\$$\indent\indent$
    \emph{База} Очевидна. $|V| = 1$
    $\\$\emph{Предположение.} Пусть утверждение верно для $\forall |V| \leq n$. Докажем утверждение для $|V| = n + 1$;
    $\\$\emph{Шаг.} Используя результат леммы выше, вершине с нулевой исходящей степенью  присваеваем номер $n + 1$. Остальной граф на $n$ вершинах занумерован по предположению индукции.
    \end{itemize}

 $\\$$\indent$ Теорема доказана.

 \paragraph{Дополнение до линейного порядка} $\\$$\indent$
 Пусть есть граф частичного порядка. Он ациклический. Тогда есть нумерация вершин по теореме выше. Дополним заданное отношение до линейного в соответствии с нумерацией. Получим линейный порядок на исходном множестве.
\end{document}